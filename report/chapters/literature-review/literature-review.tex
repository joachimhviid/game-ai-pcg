\section{Literature review}

To address the research questions regarding the optimization of dungeon layouts for "Flow," it is
necessary to examine the three intersecting domains. First we review existing PCG techniques to 
understand the limitations of current random generation. Secondly we explore Genetic algorithms
with Fitness Functions in game design, specifically analyzing how evolutionary computations computations
can be used to search for better content rather than simple construction of content. And finally
we need to analyze and understand the psycological frameworks of Flow Theory and methods to measure
player engagement, which provide the metrics requried to evaluate the success of the project.

\subsection{Procedural Content generation (PCG)}
Procedural Content Generation (PCG) refers to the algorithm in games in w hich creates game content with
limited human input \cite{Book_PCG}.  In the context of a dungeon crawler game, PCG can be primarily used 
to maximize replayability by giving the player a unique map layout and amount of enemies during 
each play session. A common method relies on constructive generation methods, these algorithms assemble
pre-authored "chunks" or rooms according to specific rules to create a coherent level structure 
\cite{Book_PCG}\cite{Constructive_PCG}.

Another unconventionel but maybe more practical solution to this could also be PCG through Machine learning on 
already existing content like levels and amount of eneimes \cite{PCG_ML}. This project tries to 
diverge from these already researched and maybe more traditional solutions by adopting 
Genetic Algorithms and Fitness functions. Rather than generating a single layout and accepting it, our
method treats level generation as an optimization problem \cite{Experience_Driven_PCG}. By generating
a population of dungeons and selecting the "fittest" ones based on engagement metrics we aim to shift
focus from simple randomness to a more Dynamic Difficult player centric design. 

\subsection{Genetic Algorithm}


\subsection{Flow Theory \& Dynamic Difficulty Adjustment (DDA) }

