\section{Literature review}

To address the research questions regarding the optimization of dungeon layouts for "Flow," it is
necessary to examine the three intersecting domains. First we review existing PCG techniques to
understand the limitations of current random generation. Secondly we explore Genetic algorithms
with Fitness Functions in game design, specifically analyzing how evolutionary computations computations
can be used to search for better content rather than simple construction of content. And finally
we need to analyze and understand the psycological frameworks of Flow Theory and methods to measure
player engagement, which provide the metrics requried to evaluate the success of the project.

\subsection{Procedural Content generation (PCG)}
Procedural Content Generation (PCG) refers to the algorithm in games in w hich creates game content with
limited human input \cite{Book_PCG}.  In the context of a dungeon crawler game, PCG can be primarily used
to maximize replayability by giving the player a unique map layout and amount of enemies during
each play session. A common method relies on constructive generation methods, these algorithms assemble
pre-authored "chunks" or rooms according to specific rules to create a coherent level structure
\cite{Book_PCG}\cite{Constructive_PCG}.

Another unconventionel but maybe more practical solution to this could also be PCG through Machine learning on
already existing content like levels and amount of eneimes \cite{PCG_ML}. This project tries to
diverge from these already researched and maybe more traditional solutions by adopting
Genetic Algorithms and Fitness functions. Rather than generating a single layout and accepting it, our
method treats level generation as an optimization problem \cite{Experience_Driven_PCG}. By generating
a population of dungeons and selecting the "fittest" ones based on engagement metrics we aim to shift
focus from simple randomness to a more Dynamic Difficult player centric design.

\subsection{Genetic Algorithm}
As outlined in the introduction, this project diverges from traditional constructive generation by treating
level design as an optimization problem. Genetic Algorithms (GA) are a subclass of evolutionary computation
inspired by biological evolution, utilizing mechanisms such as selection, crossover, and mutation to evolve
solutions to complex problems. In the context of games, Search-based Procedural Content Generation (SBPCG)
employs these algorithms to search through the space of possible content to find instances that maximize
a specific utility function.

Unlike constructive methods that assemble pre-authored chunks based on rigid rules, a GA operates by
maintaining a population of candidate dungeon layouts. In each generation, the algorithm evaluates these
candidates using a fitness function—a metric designed to quantify the quality of the content. "Fittest"
candidates are selected to breed the next generation, theoretically converging on an optimal layout that
balances specific design criteria.

While this project specifically applies GAs to dungeon crawlers, the methodology is not unique
to this genre. Significant research has demonstrated the efficacy of genetic algorithms in other
game types, such as platformers. In those contexts, GAs have been used to evolve level segments that
ensure playability and adherence to specific rhythm or difficulty curves. This cross-genre
application highlights the versatility of evolutionary computation; whether ensuring jump distances
in a platformer or room connectivity in a dungeon crawler, the core principle remains the iterative
evolution of content to satisfy a fitness function derived from design constraints.
\cite{Automatic_Game_world_with_GA} \cite{GA_in_games}

\subsection{Flow Theory \& Dynamic Difficulty Adjustment (DDA) }

To effectively utilize a Genetic Algorithm for improving player experience, it is crucial to define
the target metrics for the fitness function. This project relies on the psychological framework of Flow
Theory to guide the generation process. Flow is described as a state of optimal experience where a
player is fully immersed, occurring when there is a balance between the challenge presented by the
game and the player's skill level.

When content is generated randomly, it risks breaking this balance, resulting in layouts that are
either trivially easy or unfairly difficult. This inconsistency can lead to frustration and
disengagement. To mitigate this, this project integrates principles of Dynamic Difficulty
Adjustment (DDA). DDA systems traditionally adjust game parameters in real-time based on '
player performance; however, in this implementation, the adjustment occurs during the generation phase.
\cite{Case_for_DDA} \cite{GA_for_Real-Time_DDA} \cite{DDA_Book}

By incorporating "Experience-driven Procedural Content Generation" (EDPCG), the system attempts
to model player experience and use it as a driver for the content generation. The fitness function '
serves as the bridge between Flow Theory and the Genetic Algorithm: it evaluates dungeon layouts
not just on structural validity, but on their potential to maintain the delicate balance between
challenge and skill. This approach aims to minimize the risk of repetitive or frustrating layouts
often observed in static or purely random generation methods.
\cite{Experience_Driven_PCG}

