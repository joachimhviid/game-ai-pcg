
\section*{Appendix}

\subsection{References}

The reference style to use is the IEEE one, below we provide a short summary, for more information check out \url{https://ieeeauthorcenter.ieee.org/wp-content/uploads/IEEE-Reference-Guide.pdf} and the style guide.

The style guide adds referenced directly in the LaTeX file, but I suggest you use BibTex instead. This document is set up to provide an example of BibText usage. The main difference is that the bibliography is stored separately (see the biblio.bib file). BibTex provides various advantages compared with the example in the style guide, the main ones is that the order of references in the bib file doesn't matter (the references are properly ordered every time you compile the document) and that you if you don't cite a reference it is not included in the final document automatically. References are also formatted automatically to match the IEEE style.

Citations must be numbered consecutively within brackets according to when they are cited in the text; this is all done/updated automatically when you use the \textbackslash cite command. The sentence punctuation follows the bracket. Refer simply to the reference number, as in—do not use “Ref.” or “reference ” except at the beginning of a sentence: “Reference was the first ...”. The order of the references in the bibliography should follow the order of citation in the report, i.e. the first paper you cite will be the first in the references, the second one the second, etc. Thankfully that is done automatically by LaTeX: try to delete the citation in the literature review section and see how the numbers are updated as you recompile the project.

The reference is usually immediately after the referred theory, algorithm, author, etc. If you refer to the whole sentence or paragraph, put the reference at the end (before full stop, if it refers only to the previous sentence, after the full stop if it refers to the paragraph).

If you express somebody else’s ideas by your own words, then put the reference immediately after the idea. If you express somebody’s ideas by her/his own words, then it is a quote, remember to use quotation marks “…”!

If you want to reference to something that is not an academic paper (like a webpage) use a footnote instead. See appendix D for how to format footnotes.

\subsection{Figures and Tables}

\begin{table}[htbp]
\caption{Table Type Styles}
\begin{center}
\begin{tabular}{|c|c|c|c|}
\hline
\textbf{Table}&\multicolumn{3}{|c|}{\textbf{Table Column Head}} \\
\cline{2-4} 
\textbf{Head} & \textbf{\textit{Table column subhead}}& \textbf{\textit{Subhead}}& \textbf{\textit{Subhead}} \\
\hline
copy& More table copy$^{\mathrm{a}}$& &  \\
\hline
\multicolumn{4}{l}{$^{\mathrm{a}}$Sample of a Table footnote.}
\end{tabular}
\label{tab1}
\end{center}
\end{table}

\begin{figure}[htbp]
\centerline{\includegraphics{assets/fig1.png}}
\caption{Example of a figure caption.}
\label{fig}
\end{figure}

Figures and tables should be only at the top and bottom of columns. Avoid placing them in the middle of columns. Large figures and tables may span across both columns. Figure captions should be below the figures; table heads should appear above the tables. Insert figures and tables after they are cited in the text, or at most on the same page. Use the abbreviation “Fig. \ref{fig}”, even at the beginning of a sentence. 

Remember that you can use labels and the \textbackslash ref command to have LaTeX update the number automatically if you move figures/tables around. See the previous paragraph (in the code) for an example.

\subsection{Code snippets}

The easiest way to add code to the report is to take a screenshot and add it as if it was a figure.

Another option is to add the code directly, for example using the listings package. Here follows an example:

\begin{lstlisting}[language={[Sharp]C}]
public class Player : MonoBehaviour
{
    void Start()
    {
        
    }

    void Update()
    {
        
    }
}
\end{lstlisting}

\subsection{Footnotes}

You can add footnotes by using the \textbackslash footnote command. That will add a new footnote automatically and number it in the text like this\footnote{This is a footnote.}. Remember to use footnotes if you want to reference non-academic documents, such as links. For example, “We used a C\# machine learning variable to implement an artificial neural network\footnote{\url{https://www.ml-library.org}}.” Do not put footnotes in the abstract or reference list. Use letters instead of numbers if you need table footnotes.
