\section{Introduction}

% The purpose of this section is make the reader interested and informed in the project.

Alot of dungeon crawler games, procedurally generate their dungeon layouts, but this can result in repetitive, easy or even frustrating layouts. A
good example of this is the game called "The binding of Issac" which is a roguelike game where the dungeon layouts are procedurally generated, and 
with their downloadable content (DLC's) they have added alot of frustruating levels that are too hard for an average player to complete.

This problem can occur in other games as well, but the binding of Issac is a good example of it and leads to a lot of frustration, negative reviews
and in the end lower sales. Its important to note that this problem is not only limited to dungeon crawler games, but all types of games and its a
field thats been studied for a long time, but still not fully understood, because its a complex problem that involves a lot of different factors, including human feelings.
With new technologies, understanding of players and the help of AI, we can now start to understand this problem and create games that are more engaging and fun to play.



\begin{itemize}
	%\item
	%\item What is the problem, why is it a problem, and why is it important? (Is it something that hasn’t been studied yet? Is there an unsolved issue? Previous work presents inconclusive/contradicting results? Is there a new technology we can leverage to gain more knowledge?). One or two paragraphs.
	\item Objectives of the report: what is your actual goal? What are you hoping to achieve? About one paragraph.
	\item How do you plan on achieving your goal? This is where you want do define your research questions and hypotheses. When you define these, you are forming the basis of your project, and should be the focus of the rest of the text. If you have a very broad RQ, try to break it down in smaller, more quantifiable questions. As an example:
	\begin{itemize}
		\item H: If a genetic algorithm is configured to iteratively search for dungeon layouts that achieve a balance between 
		challenge against player skill, then the resulting procedural elves will yield a heigher engagement, create smoother flow and
		better gameplay outcomes, than a comparable static generation method.
		\item RQ1: How can a genetic Algorithm be used to generate dungeon layouts that achieve a 
		balance between challenge and skill to create a good flow in a procedural dungeon crawler?
		\item RQ2: How do procedural generation methods (Genetic Algorithm vs. static) 
		interact to influence player experience, engagement and gameplay outcomes.
	\end{itemize}
\end{itemize}


