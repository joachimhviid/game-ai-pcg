\section{Introduction}

% The purpose of this section is make the reader interested and informed in the project.

Procedural Content Generation (PCG) is a mechanic in many modern games, allowing developers to create
vast amounts of content without manual design. However while PCG offers variety, it can result in layouts
that are repetitive, easy or unfairly difficult. An example of this of this issue is observed in the rougelike game,
The Binding of Isaac \cite{TheBindingOfIsaac}. While the base game utilizes procedural generation effectively, their downloadable content (DLC)
introduced layouts and mechanics that some players found frustratingly difficult, surpassing the skill level
of the average user. 

This imbalance leads to player frustration, negative reviews and ultimately, lower commercial 
performance. The issue is not unique to The Binding of Isaac, it's a challenge across not only the genre 
but many games as well, because it involves complex interactions between game design and human factors. With
the advancement of AI and player modelling, there is an opportunity to create a system that can adapt to a players needs.

This project explores the implementation of Genetic Algorithms (GA) to generate dungeon layouts that 
tries to prioritize player engagement together with difficulty to create good flow. Building on the Minidungeons
framework by Antonios Liapis \cite{liapis_minidungeons}, we aim to configure a GA that iteratively creates 
layouts for achieving an optimal balance between challenge and player skill.

This study tries to uncover the hypothesis (H) that a genetic algorithm configured 
to balance challenge and skill will yield higher engagement and smoother gameplay flow than comparable
static generation methods.

\begin{itemize}
	\item H: If a genetic algorithm is configured to iteratively search for dungeon layouts that achieve a balance between 
	challenge against player skill, then the resulting procedural elves will yield a heigher engagement, create smoother flow and
	better gameplay outcomes, than a comparable static generation method.
	\item RQ1: How can a genetic Algorithm be used to generate dungeon layouts that achieve a 
	balance between challenge and skill to create a good flow in a procedural dungeon crawler?
	\item RQ2: How do procedural generation methods (Genetic Algorithm vs. static) 
	interact to influence player experience, engagement and gameplay outcomes.
\end{itemize}


