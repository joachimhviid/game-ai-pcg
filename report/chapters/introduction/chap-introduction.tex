\section{Introduction}

The purpose of this section is make the reader interested and informed in the project.
\begin{itemize}
	\item Background of the project: what is the problem domain? A paragraph or two to give relevant factual information about what we are talking about. Ideally this should allow non-experts to more or less understand what the report is about.
	\item What is the problem, why is it a problem, and why is it important? (Is it something that hasn’t been studied yet? Is there an unsolved issue? Previous work presents inconclusive/contradicting results? Is there a new technology we can leverage to gain more knowledge?). One or two paragraphs.
	\item Objectives of the report: what is your actual goal? What are you hoping to achieve? About one paragraph.
	\item How do you plan on achieving your goal? This is where you want do define your research questions and hypotheses. When you define these, you are forming the basis of your project, and should be the focus of the rest of the text. If you have a very broad RQ, try to break it down in smaller, more quantifiable questions. As an example:
	\begin{itemize}
		\item H: Games can help teaching math to young pupils
		\item RQ1: How can we design a game that includes math concepts while still being entertaining to young pupils?
		\item RQ2: How well does it compare with traditional educational materials?
	\end{itemize}
\end{itemize}


