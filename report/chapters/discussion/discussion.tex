\section{Discussion}

This chapter interprets the results presented previously, evaluating the effectiveness of the Genetic Algorithm (GA) in generating 
dungeon layouts optimized for Flow. We discuss the implications of using AI agents as proxies for human players, the limitations 
of the current implementation, and how this work situates itself within the broader field of Procedural Content Generation (PCG).

\subsection{Interpretation of Results}
The primary objective of this study was to validate the hypothesis that a GA configured to balance challenge and skill yields higher 
engagement potential than static generation methods. The results indicate that the system successfully evolves dungeon layouts that 
adhere to specific difficulty curves tailored to the agent's performance.

While the project relies on an AI agent rather than human participants, the "proof of concept" demonstrates that the fitness function 
effectively steers the generation process away from trivial or impossible layouts—common pitfalls in purely random generation. By treating 
level design as an optimization problem, the system ensures that the resulting content theoretically lies within the "Flow channel," matching 
the challenge to the simulated skill level. This suggests that the core mechanic of the GA is valid: it can identify and refine content 
based on engagement metrics, provided those metrics are accurately modeled.

\subsection{Limitations}
Despite promising results, there are several limitations. The most significant constraint is the reliance on AI agents to 
simulate player experience. Flow Theory is inherently psychological, relying on a player's subjective emotional state (e.g., anxiety vs. boredom). 
An AI agent, no matter how sophisticated, cannot "feel" flow, only imitate it; it can only satisfy mathematical heuristics representing difficulty. Consequently, 
while the generated levels are structurally sound and balanced for the agent, we cannot definitively claim they would induce a state of Flow in a 
human player without empirical user testing.

Additionally, the scope of the generated dungeons is relatively small. Scaling this approach to larger, more complex environments might introduce 
performance bottlenecks, as the computational cost of evaluating fitness for a larger population of complex dungeons could be prohibitive for 
real-time applications compared to faster constructive methods. 

\subsection{Future work}
To bridge the gap between theoretical balance and actual player engagement, the next logical step would be to conduct user studies. Replacing the AI 
agent with human players would provide qualitative data on enjoyment and immersion, allowing for the calibration of the fitness function based 
on real-world feedback.

Furthermore, this project implements the difficulty adjustment during the generation phase. Future research could explore "Real-Time Dynamic 
Difficulty Adjustment," where the GA runs in the background during gameplay, evolving the next segment of the dungeon based on the player's 
performance in the current room or level of a dungeon. This would create a truly adaptive experience that responds dynamically to the player's 
fluctuating skill level.

\subsection{Comparison with Existing Literature}
Our findings align with existing research in Search-based PCG, specifically the work of Liapis on MiniDungeons \cite{liapis_minidungeons}, 
which demonstrates the viability of evolutionary algorithms for level design. However, where traditional approaches often prioritize solvability 
or structural variety, our approach emphasizes the \textit{quality} of the experience through the lens of Flow Theory.

Compared to constructive generation methods \cite{Constructive_PCG}, which assemble pre-authored chunks, our GA approach offers greater 
granularity in difficulty tuning. While constructive methods are faster and computationally cheaper, they often lack the nuance to fine-tune 
a specific difficulty curve for an individual player. This project highlights the trade-off between the speed of random generation and the 
tailored quality of evolutionary computation.
