\section{Discussion}

\begin{displayquote}
    {\color{red}Note on opinions: by default, try to avoid opinions,
    you want to base everything on facts or previous
    literature. The discussion section is the exception to
    that rule: here we want to see what your interpreta-
    tion of the results is (but this should still be based
    on the results you obtained)} 

\end{displayquote}

This is the section in which you want to tell the reader what the results you obtained mean, and where you want to convince the reader that you did in fact answer your research questions.

Remember to use appropriate methods and statistical tests to provide an analysis of the results and \hl{make sure you respond to each research question/hypothesis!}

Tell the reader what is bad about your study (are there some limitations? Maybe you didn’t have enough participants to have some strong results? Maybe your participants are not representative of the target population? Maybe you realized the testing methods you used have some pitfalls?) How do you expect these limitations might affect your results?

What is good about your study? (Are there some strengths to it? Did you do something particularly smart? Did you use some innovative equipment?)

Can your results compare to existing literature? Why yes/no? If yes, how do they compare? Limit this to a brief summary of how your results fit with the existing knowledge.

