
\section{Method/Implementation}

\subsection{Iteration 1: Baseline Setup}
\subsubsection{About}
The initial iteration of this project involved creating a baseline setup based on Antonios Liapis' MiniDungeons project \cite{liapis_minidungeons} ported to Python by ganyariya \cite{GymMd}.
This iteration was used to provide a testbed for evalutating the quality of levels generated by the future level generator. MiniDungeons was used as a testbed as a means of
limiting the scope of the project to generating levels, rather than creating a game \textit{and} levels. 

The Python port of MiniDungeons (\texttt{gym-md}) \cite{GymMd} proved to be a sufficient starting point for this projects goals, however it was created using a legacy version of Gymnasium \cite{Gymnasium}\cite{towers2025gymnasiumstandardinterfacereinforcement}.
For that reason, in order to avoid issues with dependencies, a fork of \texttt{gym-md}\cite{GymMdFork} was created with the express purpose of updating its core dependencies to allow further development.

The final step of this iteration was to setup the project repository to be able to run and modify MiniDungeons from a central location. The project was created using Poetry\cite{PythonPoetry} to manage dependencies and standardize entry points for the code.
The project was setup to extend \texttt{gym-md} while still allowing for new levels to be used in the environment. The iteration concluded with a standard entry point for running
MiniDungeons with hardcoded custom dungeon layouts.

\subsubsection{Evaluation}
evaluate

\subsubsection{Reflections and decisions}
reflect and decide

% -------------------------------------------------------------------

\subsection{Iteration 2: Refactored Environment and Agent}
\subsubsection{About}
Second iteration involved refactoring the MiniDungeons environment.

Purpose: Improve our ability to interact with the environment.

Work:
\begin{itemize}
	\item Recreated \texttt{gym-md} environment and moved rendering to pygame.
	\item Create a deterministic agent based on MiniDungeon personas \cite{holmgard2014evolvingpersonas}.
\end{itemize}

\subsubsection{Evaluation}
evaluate

\subsubsection{Reflections and decisions}
reflect and decide

% -------------------------------------------------------------------

\subsection{Iteration 3: Dungeon Generator}
\subsubsection{About}
Third iteration involved procedurally generating dungeon layouts.

\subsubsection{Evaluation}
evaluate

\subsubsection{Reflections and decisions}
reflect and decide

% \subsection{Iteration X}

% You will often go through various iterations of product/game design, duplicate this subsection for each iteration before the final one. 

% \begin{itemize}
% 	\item Usually, iteration 1 would be a low fidelity prototype.
% 	\item Iteration 2 could be a high-fidelity prototype (vertical slice or functional prototype).
% 	\item Iteration 3 or later are refinements (extended functionality, improvements/polish, usability tweaks, bug-fixing) or the final product.
% \end{itemize}

% \subsubsection{About}

% Describe what this iteration was all about.

% \begin{itemize}
% 	\item What is the purpose of this iteration?
% 	\item What were the goals?
% 	\item Describe and justify design decisions. For example, “we implemented feature X because it relates to gamification concept Y”, or “in the previous iteration we identified issues with the user interface, so we reworked by changing X, Y, and Z.”
% 	\item Include images/sketches. 
% \end{itemize}

% \subsubsection{Evaluation}

% Describe how you evaluated this iteration, for example what kind of user testing or technical testing you conducted. This should include:

% \begin{itemize}
% 	\item Test design: explain what is the objective of the study: what kind of information are you trying to obtain? What kind of study did you conduct? (Usability test, playtest, something more related to your RQs, etc.?) Did you use a specific established methodology? Usually this is not a very large part and serves as an introduction to your test.
% 	\item Test methodology: this is usually a larger part where you need to provide all the details of how you conducted your test. 
% 	\begin{itemize}
% 		\item How did you structure the test? For example, “The participant played the game while doing think-aloud, then compiled a survey.”
% 		\item Ethical reflections: is there some problematic aspect that you had to address? If so, how did you address it? How are you compliant with GDPR laws? 
% 		\item Technical details: where did you conduct the study? What hardware was used? 
% 		\item Detailed information about the test. For example, if you used a survey, what questions were included? If you conducted interviews, how did you structure them? How did you collect the data?
% 	\end{itemize}
% 	\item Results of the test
% 	\begin{itemize}
% 		\item How many participants did you have?
% 		\item What are some basic demographics for the participants?
% 		\item Present the data collected (use tables and graphs, rather than text). 
% 		\item Make sure you mention what statistical methods you use (and why).
% 	\end{itemize}
% \end{itemize}

% \subsubsection{Reflections and decisions}
% In this section describe how are you using the results you described in the previous section to further your development, to answer your research questions, or in general to evaluate your project (this last part is mostly relevant for the final iteration)

% \begin{itemize}
% 	\item How do you interpret the results you just described? What do they mean?
% 	\item How are you planning to use these in the following iteration?
% \end{itemize}

% \subsection{Final iteration}

% This section describes the final tweaks you applied from the previous iteration. Show the main differences and describe the features of the final product (just like the “About” section in the previous iterations)
